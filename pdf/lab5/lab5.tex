\documentclass[12pt]{article}

\usepackage{amsmath,amsthm,amssymb}
\usepackage{hyperref}
\usepackage{listings}
\lstset{
	columns=[c]fullflexible,
	emph={},
	emphstyle=\rmfamily\textit,
	tabsize=4,
	numbers=none,
	numberstyle=\scriptsize,
	stepnumber=1,
	escapeinside={(*}{*)},
	mathescape=true,
	sensitive=false,
    frame=none,
    keepspaces,
    extendedchars=true,
    showspaces=false,
    showstringspaces=false,
}

\usepackage{tikz}
\usetikzlibrary{calc,shapes,decorations.pathmorphing,snakes,arrows,automata,positioning,patterns,chains,shapes.multipart, arrows.meta}

\let\origthelstnumber\thelstnumber
\makeatletter
\newcommand*\Suppressnumber{%
  \lst@AddToHook{OnNewLine}{%
    \let\thelstnumber\relax%
     \advance\c@lstnumber-\@ne\relax%
    }%
}
\newcommand*\Setnumber[1]{%
  \setcounter{lstnumber}{\numexpr#1-1\relax}%
  \lst@AddToHook{OnNewLine}{%
   \refstepcounter{lstnumber}
  }%
}
\newcommand*\Reactivatenumber[1]{%
  \setcounter{lstnumber}{\numexpr#1-1\relax}
  \lst@AddToHook{OnNewLine}{%
   \let\thelstnumber\origthelstnumber%
   \refstepcounter{lstnumber}
  }%
}
\makeatother


\begin{document}
	\title{BMC for Peterson}
	\author{Alfons Laarman}
	
	\maketitle

The aim of this assignment is to encode the Peterson mutex protocol
into propositional logic and perform Bounded Model Checking using
a SAT solver.

Install the following tools:
\begin{description}
\item[bmc] ~\\\texttt{git clone {https://github.com/alaarman/bmc.git}}\\
			\texttt{cd bmc\\make}
\item[minisat] ~\\\texttt{wget http://minisat.se/downloads/minisat-2.2.0.tar.gz}\\
				\texttt{tar zxvf minisat-2.2.0.tar.gz}\\
				\texttt{cd minisat\\export MROOT="`pwd`"}\\
				\texttt{cd core\\gmake rs}\\
				\texttt{export PATH="\$PATH:\$MROOT/core"}
\end{description}





We will use the C preprocessor (\texttt{cpp}) to simplify generation of long of
long formulas.

We do not generate CNF directly. Instead, the preprocessor generates 
a simple propositional formula, which is converted to CNF with the \texttt{bool2cnf} tool.
An propositional formula consists of variables and a collection of Boolean connectives.
The syntax of propositional formula is as follows.

\begin{lstlisting}
expr ::
atom (*\hfill*);; a variable identifier [0-9a-zA-Z_]+
| "!" expr (*\hfill*);; logical not
| expr1 "&" expr2 (*\hfill*);; logical and
| expr1 "|" expr2 (*\hfill*);; logical or
| expr1 "->" expr2 (*\hfill*);; logical implication
| expr1 "=" expr2 (*\hfill*);; logical equivalence
\end{lstlisting}

The order of parsing precedence from high to low is
\begin{lstlisting}
!
&
|
->
=
\end{lstlisting}
Operators of equal precedence associate to the left. Parentheses may be used to group expressions.

\begin{figure}
{
\lstset{
	numbers=left,
	firstnumber=0
}
\begin{tabular}{p{.1cm}p{5cm}||p{.1cm}p{3.5cm}}
&
\begin{minipage}{14em}
\begin{lstlisting}
	flag[0] = 1;
	turn = 1;
	!flag[1] || turn==0;(*\Suppressnumber*)
	/* Critical section */(*\Reactivatenumber{3}*)
	flag[0] = 0; goto 1;
\end{lstlisting}
\end{minipage}
&&
\begin{minipage}{14em}
\begin{lstlisting}
	flag[1] = 1;
	turn = 0;
	!flag[0] || turn==1;(*\Suppressnumber*)
	/* Critical section */(*\Reactivatenumber{3}*)
	flag[1] = 0; goto 1;
\end{lstlisting}
\end{minipage}\\
\end{tabular}
}
\caption{Peterson with critical section at Line $3$}
\label{fig:peterson}
\end{figure}


\begin{proof}[Task 1]
~\\
\begin{enumerate}
	\item 
Complete the encoding of the following two-process Peterson mutex protocol from \autoref{fig:peterson}
in \texttt{peterson.enc.c}.
	\item 
Run \texttt{bmc.sh} to verify the system.
	\item 
Introduce an error in the condition on Line 2. \texttt{bmc.sh} should print a counterexample.
	\item
\emph{\bf Hand in your both the correct and incorrect encoding plus a printout counterexample generated by \texttt{bmc.sh}}
\end{enumerate}


\end{proof}



\begin{proof}[Task 2 (optional)]
	Do the same for \texttt{peterson4.enc.c}, which is based on \label{fig:peterson4}.
\end{proof}



\begin{figure}
\begin{center}
\tikzset{myellipse/.style={ellipse,draw,minimum width=1.3cm,minimum height=5mm},
myrectangle/.style={text width=2.4cm,align=center,midway,font=\scriptsize},
branode/.style={midway,font=\scriptsize}}
\scalebox{1.2}{
\begin{tikzpicture}[>=stealth, node distance=2cm]
\node[myellipse,ultra thick](1){0};
\node[myellipse, node distance=3cm,below left=of 1](2){1};
\node[myellipse, node distance=3cm,below right=of 1,ultra thick](3){2};
%\node[myellipse, node distance=3cm,right=of 3](4){3};
\node[myellipse, node distance=2cm,above right=of 1](5){3};


  \draw[<-] (1) -- node[above] {\scriptsize [ $level[i] := 0$ ]} ++(-3cm,0);

\draw[->](1)--(2)node[myrectangle,left=1mm]{$level[i] < N-1$};
\draw[->](2)-- node[myrectangle,below=1mm]{$last[level[i]] := i$} (3);
\draw[->](2)--(3)node[myrectangle,above=1mm]{[ $k := 0$ ]};
\draw[->,loop right](3) to node[myrectangle, above=2mm]
	{\begin{minipage}{4cm}\centering
		$k < N \land$ \\
		$(k=i \lor last[level[i]] \neq i\lor$\\
		$\phantom{X}level[k] >= level[i])$
	\end{minipage}
	} node[myrectangle, right=-8mm, yshift=-3mm]{[ $k$++ ]} (3);
%\draw[->,bend right=20](4) to node[myrectangle, above=1mm]{$k++$} (3);
\draw[->](3)  to node[myrectangle,sloped,above=1mm]{[ $level[i]$++ ]} 
				 node[myrectangle,sloped,below=1mm]{$k\ge N$} (1) ;
\draw[->,bend right=20](5) to node[myrectangle,sloped,above=1mm]{[ ${level[i] := 0}$ ]} (1);
\draw[->,bend right=20](1) to node[myrectangle,sloped,above=1mm]{} (5);
\end{tikzpicture}
}\end{center}
\caption{Finite State Machine (FSM) of Peterson thread $i$ with critical section at Location $3$}
\label{fig:peterson4}
\end{figure}



	
\end{document}}