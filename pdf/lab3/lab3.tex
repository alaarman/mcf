\documentclass[12pt]{article}

%!TEX root = main.tex

% for subfigures
\usepackage[textfont={it,scriptsize}]{subfig}

\usepackage[vlined, boxed,linesnumbered,noend]{algorithm2e}

\usepackage{cprotect}
\usepackage{booktabs}
\let\proof\relax
\let\endproof\relax
\usepackage{amsthm}
\usepackage{amsmath,amssymb,mathtools,stmaryrd}
%\usepackage{latexsym}
\usepackage{graphicx}
\usepackage{complexity}
\usepackage{tikz}
\usepackage{xspace}
%\usepackage{verbatim}
\usepackage{multicol}
%\usepackage{fancybox}
\usepackage{wrapfig}
\usepackage{enumitem}
%\usepackage{paralist}
\usepackage[bottom]{footmisc}
\usepackage{footnote}
\makesavenoteenv{table}
\makesavenoteenv{tabular}
%\usepackage{pbox}
%\usepackage{float}
\usepackage[font={scriptsize,it}]{caption}
%\usepackage[font={scriptsize,it}]{subcaption}
\usepackage{adjustbox}
\usepackage{color}
% For striking out text:
\usepackage{cancel}
\usepackage[normalem]{ulem}
%\usepackage[nounderscore]{syntax}


% For Arxiv:
\usepackage{ifpdf}
    \ifpdf
\usepackage[colorlinks,citecolor=blue!60!black!70,linkcolor=blue!60!black!70]{hyperref}
    \else
%      do something for regular latex or pdflatex in dvi mode
    \fi


\usetikzlibrary{automata}
\usetikzlibrary{positioning}
%\usetikzlibrary{decorations.pathmorphing}
\usetikzlibrary{arrows,patterns}
\usetikzlibrary{calc}
\usetikzlibrary{decorations.pathreplacing,decorations.markings}


\renewcommand\paragraph[1]{\vspace{.0ex}\par\noindent\textbf{#1}.\;\;}
\newtheorem{observation}{Observation}
\newtheorem{alg}{Algorithm}

\renewcommand{\topfraction}{0.9}
\renewcommand{\bottomfraction}{0.9}


\def\sectionautorefname{Sec.}
\def\subsectionautorefname{Sec.}
\def\corollaryautorefname{Cor.}
\def\definitionautorefname{Def.}
\def\figureautorefname{Fig.}
\def\exampleautorefname{Ex.}
\def\algorithmautorefname{Alg.}
\def\appendixautorefname{App.}
\def\algautorefname{Alg.}
\def\observationautorefname{Obs.}
\def\lemmaautorefname{Lemma}
\def\theoremautorefname{Th.}
\def\equationautorefname{Eq.}
\newcommand{\subfigureautorefname}{\figureautorefname}


\let\ssaphi=\phi % Georg: I need the original phi
\let\phi=\varphi 
\let\epsilon=\varepsilon
\let\code=\verb
\renewcommand\implies{\Rightarrow}


\hyphenation{bet-ween}
\hyphenation{da-ta-base}
\hyphenation{wor-ker}



%%% Local Variables: 
%%% mode: latex
%%% TeX-master: "main"
%%% End: 

%!TEX root = main.tex


\newcommand\defaccr[2]{\newcommand#1{#2\xspace}}
\newcommand\defmath[2]{\newcommand#1{\ensuremath{#2}\xspace}}
\newcommand\concept[1]{\textit{#1}}



\newcommand\ccode[1]{\texttt{#1}}


\newcommand{\overarrowi}[1]{\xrightarrow{#1}}
\newcommand{\overarrow}[1]{
  \mathchoice{\raisebox{-3pt}{ $\overarrowi{#1}$ }}
             {\raisebox{-3pt}{ $\overarrowi{#1}$ }}
             {\raisebox{-3pt}{ $\overarrowi{#1}$ }}
             {\raisebox{-3pt}{ $\overarrowi{#1}$ }}}

%containers
\providecommand{\tuple}[1]{\ensuremath{\left( #1 \right)}}
\providecommand{\set}[1]{\ensuremath{\left\lbrace #1 \right\rbrace}}
\providecommand{\sequence}[1]{\ensuremath{\left( #1 \right)}}
\providecommand{\sizeof}[1]{\ensuremath{\left\vert{#1}\right\vert}}
\providecommand{\vect}[1]{\ensuremath{( \begin{matrix} #1 \end{matrix} )}}
\providecommand{\always}[1]{\ensuremath{\left[ #1 \right]}}
\providecommand{\possibly}[1]{\ensuremath{\left\langle #1 \right\rangle}}
%\newcommand{\powerset}[1]{\wp({#1})}
\newcommand{\powerset}[1]{\ensuremath{\mathbf{2}^{#1}}}
% Semantics
\newcommand{\eval}[2][]{\ensuremath{\llbracket #2\rrbracket^{#1}}}

% Map (bijection)
\DeclareMathOperator{\bijmap}{%
 \rlap{\ensuremath{\rightarrowtail}}%
 	{\ensuremath{\mkern2mu\twoheadrightarrow}}}

% Abs, Floor, Ceil
\providecommand{\abs}[1]{\lvert#1\rvert}
\providecommand{\floor}[1]{\lfloor#1\rfloor}
\providecommand{\ceil}[1]{\lceil#1\rceil}

\newcommand{\defn}{\,\triangleq\,}

% Rotate text
\newcommand{\turner}[3][10em]{% \turn[<width>]{<angle>}{<stuff>}
  \rlap{\rotatebox{#2}{\begin{varwidth}[t]{#1}#3\end{varwidth}}}%
}


\providecommand{\Assign}[2]{{#1{~:=~}#2}}
\providecommand{\AssignState}[2]{\State \Assign{#1}{#2}}


%initial state
\newcommand{\inits}{s^0}

%transition relation expression
\newcommand{\transrelexpr}{\mathsf{next}}%\mathord{\hookrightarrow}}

\defmath{\explored}{\mathsf{explored}}
\defmath{\visited}{\mathsf{visited}}
\defmath{\level}{\mathsf{level}}
\defmath{\tmp}{\mathsf{tmp}}
\defmath{\WM}{\mathsf{WM}}
\defmath{\RM}{\mathsf{RM}}
%next state small caps
\renewcommand{\next}{\mathsf{next}}

%next state small caps
\newcommand{\nextstate}{\textsc{next-state-r2w}\xspace}

%!TEX root = main.tex


\defmath\dna{\ensuremath\mathcal{\not\bowtie}}
\defmath\B{\ensuremath\mathit{B}}

\newcommand\en{\ensuremath{\mathit{en}}\xspace}
\newcommand\modell{\ensuremath{M_{\structuraltransitions}}\xspace}
\newcommand\modelr{\ensuremath{\modell^R}\xspace}
\newcommand\buchi{\ensuremath{\mathbb{B}}\xspace}
\newcommand\accepting{\ensuremath{\mathcal{F}}\xspace}
\newcommand\cross{\ensuremath{\modell\otimes \buchi_{\neg\varphi}}\xspace}
\newcommand\crossr{\ensuremath{\modell^R\otimes \buchi_{\neg\varphi}}\xspace}

\newcommand\vis{\ensuremath{\structuraltransitions_\mathrm{v}}\xspace}
\newcommand\vismax{\ensuremath{\vis^\mathrm{min}}\xspace}
\newcommand\vislabels{\ensuremath{\vis^\mathrm{nes}}\xspace}
\newcommand\visdyn{\ensuremath{\vis^\mathrm{dyn}}\xspace}

\renewcommand\implies{\Rightarrow}

\newcommand{\instantiator}{\textsc{Instantiator}\xspace}
\newcommand\nips{\tool{Nips}}
\newcommand\nipsvm{\nips\-\tool{VM}\xspace}
\newcommand\spin{{\sc Spin}\xspace}
\newcommand\spinja{{\sc SpinJa}\xspace}
\newcommand\promela{\textsc{Promela}\xspace}
\newcommand\dve{\textsc{Dve}\xspace}
\newcommand\beem{\textsc{BEEM}\xspace}
\newcommand\divine{\tool{DiVinE}\xspace}
\newcommand\mcrl{\ensuremath{\mu\mathrm{CRL}}\xspace}
\newcommand\mcrltwo{mCRL2\xspace}
%\newcommand\pins{\iface{Pins}\xspace}
\newcommand\pinspins{\iface{Pins2Pins}\xspace}
\newcommand\cadp{CADP\xspace}
\newcommand\ltsmin{{\sc LTSmin}\xspace}

\newcommand{\statevec}[1]{\ensuremath{\langle{#1}\rangle_{\mathrm{SV}}}}
\newcommand{\intvec}[1]{\ensuremath{\langle{#1}\rangle_{\mathrm{IV}}}}
\newcommand{\foldedvec}[1]{\ensuremath{\langle{#1}\rangle_{\mathrm{FV}}}}
\newcommand{\pair}[1]{\ensuremath{\langle{#1}\rangle}}

\newcommand{\Nat}{\ensuremath{\Bbb N}}
\newcommand{\Bool}{\ensuremath{\Bbb B}}
\DeclareMathOperator{\hash}{Hash}
\newcommand{\statepart}[1][s]{#1}

\newcommand{\struct}[1]{\ensuremath{\langle#1\rangle}}

\newcommand\logor{\vee}
\newcommand\logand{\wedge}
\newcommand{\var}[1]{\ensuremath\mathit{#1}}

\newcommand\true{\ensuremath\textbf{true}}
\newcommand\false{\ensuremath\textbf{false}}
\newcommand\guards{\ensuremath\mathcal{G}}
\newcommand\allguards{\ensuremath{\guards_{\structuraltransitions}}}
\newcommand\initialstate{\ensuremath{s_0}}
\newcommand\structuraltransitions{\ensuremath\mathcal T}
\newcommand\semantictransitions{\ensuremath\Delta}
\newcommand\states{\ensuremath{S}}
\newcommand\modelstates{\ensuremath {S_{\structuraltransitions}}}
\newcommand\model{\ensuremath{M_{\structuraltransitions} = (\modelstates,\initialstate,\semantictransitions)}}
\newcommand\structuraltransition{\ensuremath{t = (\guards,a)}}
\newcommand\stateelements{\ensuremath{E_1 \times \ldots \times E_n}}
\newcommand\testset{\ensuremath{\mathit{Ts}}\xspace}
%\newcommand\readset{\ensuremath{\mathit{Rs}}\xspace}
%\newcommand\writeset{\ensuremath{\mathit{Ws}}\xspace}
\newcommand\variableset{\ensuremath{\mathit{Vs}}\xspace}
\newcommand\MC{\ensuremath{\mathit{MC}}\xspace}
\newcommand\findnes{\ensuremath{\mathit{find\_nes}}\xspace}
\newcommand\persistentset{\ensuremath{\mathcal{T}}\xspace}
\newcommand\stubbornset{\ensuremath{\mathcal{T}_s}\xspace}
\newcommand\workset{\ensuremath{Q}\xspace}
\newcommand\keyset{\ensuremath{\mathcal{T}_{\mathit{k}}}\xspace}
\newcommand\nkset{\ensuremath{\mathcal{T}_{\mathit{n}}}\xspace}
%\newcommand\nes{\ensuremath{\mathcal{N}}\xspace}
\newcommand\nesmax{\ensuremath{\nes^\mathrm{min}}\xspace}
\newcommand\nespins{\ensuremath{\nes^\textsc{pins}}\xspace}
\newcommand\nds{\ensuremath{\overline{\nes}}\xspace}
\newcommand\mds{\ensuremath{\mathcal{M}}\xspace}
\newcommand\ndsmax{\ensuremath{\nds^\mathrm{min}}}
\newcommand\ndspins{\ensuremath{\nds^\textsc{pins}}}

\newcommand\trans\structuraltransitions
\newcommand\transitions\structuraltransitions


%%% Local Variables: 
%%% mode: latex
%%% TeX-master: "main"
%%% End: 


\title{Binary Decision Diagrams}

\author{{Alfons Laarman, Pim Wijn, Richard Huybers}}


\pagestyle{plain}

\def\ltsmin{\textsc{LTSmin}\xspace}


\begin{document}

\maketitle


\todo[inline]{
Simpler version first (no learning)
}

\todo[inline]{
Fix references to lectures? Split in two parts?
}

\section{Introduction}

The current lab assignment consists of two parts.


In the first part, we will use 
Decisions Diagrams (BDDs and MDDs), or DDs, to
implement reachability in a model checker based on
a partitioned next-state interface (PINS).
The performance of the implementation will be compared
to that the performance of the enumerative model
checker storing states in a tree table
in terms of state compression and runtime.


In the second part of the lab, we will assess the correctness
of BDD packages using the proof assistant Dafny~\cite{dafny}.
A proof assistant, or theorem prover, provides a way to 
construct machine checkable proofs.
The correct  implementation of BDD packages is a tricky and time consuming.
Therefore, instead, we will reason over their implementation at
a higher level using the proof assistant.



\section{BDD-Based Symbolic Model Checking}


In the previous lab, 
PINS was used with long state vectors, i.e.,
containing all program variables
(The function \texttt{GBgetTransitionsAll} returns the successors states for all
transitions and \texttt{GBgetTransitionsLong} can query the successors for a
single action or \concept{group}).
PINS also provides the next-state functions for projects, short vectors
(\texttt{GBgetTransitionsShortR2W} is like \texttt{GBgetTransitionsLong},
but takes a short vector and yields short, projected successors).

To provide information on the variables in the short vectors,
PINS provides two boolean matrices: a read matrix $\RM$ and a write matrix $\WM$.
Each row $\RM_i$ / $\WM_i$ in the matrix represents the $i$th action (or group in PINS
terminology) in the program. Each column $j$ represents a variable $x_j$
in the program. So if $\RM_{i,j}=\true$ then action $i$ reads from variable $x_j$.
We will refer to state vector over all variables $\set{x_1,...,x_n}$ as
\emph{long vectors}. The \emph{sort vectors}, on the other hand, involve only 
a subset of the variables, e.g., a \emph{read-projected} short vector for
group $i$ is defined over $\set{x_j\mid \RM_{i,j}}$, whereas
a \emph{write-projected} short vector for
group $i$ is defined over $\set{x_j\mid \WM_{i,j}}$.

In the following, let:
\begin{itemize}
	\item $\tuple{s_0, S, \hookrightarrow}$ be the transition system, 
	\item $K$ be the number of actions,
	\item $\hookrightarrow \defn \bigcup_{i\in [1..K]} \hookrightarrow_i$ be its partitioning into actions,
	\item  $S = x_1 \times ... \times x_L$ be the syntactic state space,
	\item  $W^{i} = \prod_{j \in \WM_{i}} x_j$ be the space of write-projected short vectors for action $i$, and
	\item  $R^{i} = \prod_{j \in \RM_{i}} x_j$ be the space of read-projected short vectors for action~$i$.
\end{itemize}

We write a super script $r$/$w$ to indicate that a vector is
read/write projected short vector, e.g., $s^r\in R^i$.
Existential quantification over variables (not in the projection) is used to
transit between short and long vectors:
\begin{itemize}
	\item $\rho_i(s) \defn \exists X\colon s$ for $X=\set{x_j\mid \neg \RM_{i,j}}$
\hfill (read projection)
	\item $\omega_i(s) \defn \exists X\colon s$ for $X=\set{x_j\mid \neg \WM_{i,j}}$
\hfill (write projection)
	\item $\rho^{-1}_i(s^r) \defn s'$ s.t. $s' \in S \land \rho(s')=s^r$
\hfill (inverse read projection)
	\item $\omega^{-1}_i(s^w) \defn s'$ s.t. $s' \in S \land \omega(s')=s^w$
\hfill (inverse write projection)
\end{itemize}

\def\vx{\vec x}
\begin{example}
Note that the inverse projection inserts random values for vars.

\noindent
	$\vx := \possibly{x,y,z}$\hfill (variables)\\
	$R^1 := \possibly{x,y}$\hfill (read projection of Action 1)\\
	$W^1 := \possibly{y}$\hfill (write projection of Action 1)\\

\noindent
Let $A := \set{\tuple{\textcolor{orange}{1,2},3}, \tuple{\textcolor{orange}{1,2},9}, \tuple{\textcolor{orange}{0,2},3}, \tuple{\textcolor{orange}{1,0},3}}$\\
\noindent
	$\rho_1(A) = \set{\tuple{1,2}, \tuple{0,2}, \tuple{1,0}}$\\
\noindent
	$\rho_1^{-1}(\rho_1(A)) = \{\tuple{1,2,a}, \tuple{0,2,b}, \tuple{1,0,c}\}$ for 
	some $a,b,c \in z$
\qed
\end{example}




\begin{table}[h]
\begin{tabular}{|l|l|l|}
\hline
\bf Function & \bf Type & \bf Definition \\
\hline
	\texttt{GBgetTransitionsAll(..$,s,$..)} & $S\rightarrow 2^S $  
		& \set{t\mid s\hookrightarrow t} \\ 
	\texttt{GBgetTransitionsLong(..$,s,i,$..)}& $S\times \mathbb{N} \rightarrow 2^S$
		& \set{t\mid s\hookrightarrow_i t} \\ 
	\texttt{GBgetTransitionsShortR2W(..$,s^r,i,$..)}& $R^i\times \mathbb{N} \rightarrow 2^{W^i}$
		& \set{\omega_i(t)\mid \rho_i^{-1}(s)\hookrightarrow_i t} \\ 
\hline
\end{tabular}
\end{table}


\begin{algorithm}[h!]
\SetAlgoLined
\small
\KwData{$\mathit{\RM}$,$\mathit{\WM}$,$K$, $\inits$}
\KwResult{$\visited$ \hfill (the reachable states)}
\vspace{1.5ex}$\visited \gets \{\inits\}$\\
$\level  \gets \{\inits\}$\vspace{1.5ex}\\
\For{$1 \leq i \leq K$} {
    $\explored_i^r \gets \emptyset$\\
    $\tmp_i^r \gets \emptyset$\\
    $\transrelexpr_i^{r,w} \gets \emptyset$\hfill (\textcolor{orange}{empty} r/w projected transition relations)\\
}\vspace{2ex}
\While{$\textcolor{orange}\level \neq \emptyset$\vspace{1.5ex}} {
    \For{$1 \leq i \leq K$} {
        $\tmp_i^r \gets \rho_i(\textcolor{orange}\level) \setminus \explored_i^r$\hfill (new read inputs)\\
        \For{$s^r \in \tmp_i^r  $} {
            $\transrelexpr_i^{r,w} \gets \transrelexpr_i^{r,w} \cup \{(s^r,d^w) \mid
d^w \in  \nextstate_i(s^r)\}$\\ 
        }
        $\explored_i^r \gets \explored_i^r \cup \tmp_i^r$ \\
    }\vspace{2ex}
    $\tmp \gets \emptyset$\\
    \For{$1 \leq i \leq K$} {
        $\tmp \gets \tmp \cup \textsc{Next}(\textcolor{orange}\level,
\transrelexpr_i^{r,w}, \mathit{\RM}_i, \mathit{\WM}_i)$ \label{l:next}\\
    }
    $\textcolor{orange}\level \gets \tmp \setminus \visited$\\
    $\visited \gets \visited \cup \tmp$\\
}
\caption{\textsc{reach-bfs-rw}}
\label{alg:reach-bfs}
\end{algorithm}

Remember that GBgetTransitionsAll et al. utilize callbacks to return successor states.
For briefness, in this text, we abstract from the callbacks and pretend the functions
return a set of state vectors (short or long).
we have the following overview of PINS functionality (see \texttt{pins/pins.h} for details):


\autoref{alg:reach-bfs} implements symbolic reachability in PINS
by learning the symbolic transition relation on-the-fly.
It starts with one empty DD $\next^{r,w}_i$ to store (cache)
the transition relation of action $i$.
$\level$ is a DD representing the new vectors that need to be
expanded in the main search loop (\textbf{while}).
Before the actual expansion with the \textsc{Next} operation 
on \autoref{l:next} according to Slide 45 of Lecture 5,
the transition relation is learnt (see the first \textbf{for} loop).
For each group the states in $\level$ are projected to the
relevant read variables. If new inputs $s^r$ are found for these
read values (by comparing with short vectors already in $\explored^r_i$),
the $\nextstate$ function is called to compute all successors
$d^w$ of $s^r$. The pairs \tuple{s^r,d^w} are then added to
 $\next^{r,w}_i$.

Note that PINS implements $\nextstate$ with \texttt{GBgetTransitionsShortR2W}.

To facilitate the implementation of this algorithm, we use the
VSET library (see \texttt{vset/vector\_set.h} for details).
It is wrapper around multiple DD packages. It also handles some
logic to reason over the variable domains:
\texttt{vdom\_t} represents the domain of all variables (in long vectors)
\texttt{vset\_t} represents a DD over state variable (short and long vectors),
and \texttt{vrel\_t} represents DDs for transition relations
over short or long vectors (contain unprimed and primed versions of the variables).
The functions declared in \texttt{vset/vector\_set.h} should be
self-explanatory. Note however that some overwrite one of the arguments
with the result, e.g., executing \texttt{vset\_union(vset\_t dst,vset\_t src)}
results in \texttt{dst} containing the union of the two original arguments.

The \textsc{Next} operation is implemented with \texttt{vset\_next}.

\begin{proof}[Task 1]
Implement \autoref{alg:reach-bfs} in \texttt{alg/sym-bfs.c}.
\end{proof}

\begin{proof}[Task 2]~\\
\vspace{-2em}
\begin{itemize}
	\item 
Run your algorithm on several inputs from the \texttt{examples} dir
(command: \texttt{./mc examples/x.pr --sym}).

\item
Change the BDD implementation to an MDD implementation by
setting \texttt{VSET\_Sylvan} to \texttt{VSET\_LDDmc} in 
the \texttt{init\_domain} function.
Compare the results.

\item
Compare the runs with the results from the enumerative algorithms with
tree table
(implemented in \texttt{alg/enum-dfs.c} and executable with the command:
\texttt{./mc examples/x.pr}, i.e., without \texttt{--sym}).

\item
Make a table comparing state count (should be equal across different algorithms and
DD packages), runtimes and memory use for at least three different models
and the three different approaches: tree compression, BDDs and MDDs.
\end{itemize}
\vspace{-2em}
\end{proof}

The relevant variables for (may) read and (may) write projections
come from static analysis. Static analysis is an efficient
method for overestimating which variables are involved in
which statements.
Compilers use static analysis for optimizing code
(therefore, ``efficient'' really means quadratic, at most cubic, in the size of the input program, i.e., of negligible complexity compared to the model checking operation).
To get an impression of static analysis,
we will derive read and write variables for the Peterson model, by hand.

It is trivially correct to mark all variables $j$ in the
state vector as may-read and may-write
for all actions $i$. However, this will not yield an efficient
symbolic model checking procedure, as all DDs in \autoref{alg:reach-bfs}
will include all variables.
The following definitions allow us to derive better read / write
dependencies:
\begin{definition}[Read independence]

	An action $i\in [1..K]$ is \emph{read independent} of slot $j \in [1..L]$
	when for all $s \hookrightarrow_j t$, either:
	\begin{itemize}[leftmargin=5mm]
		\item $\forall a\in x_j \colon s[j := a] \hookrightarrow_j t[j := a]$
			\hfill {(slot $j$ is irrelevant and copied)}
		\item $\forall a\in x_j \colon s[j := a] \hookrightarrow_j t$
			\hfill {(slot $j$ is irrelevant and overwritten with $t[j]$)}
	\end{itemize}
\end{definition}

\begin{definition}[Write independence]
	An action $i\in [1..K]$ is \emph{write independent} of slot $j \in [1..L]$
	when for all $s \hookrightarrow_j t$, then
	$s[j] = t[j]$ {(slot $j$ is copied)}.
\end{definition}


\begin{proof}[Task 3]
Implement the functions 
\texttt{write\_matrix} and \texttt{read\_matrix} 
in\\ \texttt{peterson/peterson.c} more precisely,
so that it can be used in symbolic model checking.

See \texttt{label\_matrix} for an example of how to implement these
functions. (The label matrix $L_{i,j}$
says which slots $j$ of the state vector are
read to evaluate a state label $i$.)
\end{proof}


\section{BDD Correctness}

For the aims of this lab, Dafny provides two advantages
over other theorem provers:
\begin{itemize}
	\item It allows us to write algorithms in a simple imperative 
	language (other theorem provers offer only non-von Neumann languages,
	i.e., functional languages, that are easier to reason about
	mathematically, but more remote from the algorithms we study).
	The verification approach of Dafny can be though of in terms of
	the Curry-Howard correspondence (programs are proofs).
	In practice this means that if we aim to prove that the output
	of a program satisfies a certain property, then we
	have to prove that every computational path through the
	program satisfies it.
	\item Dafny lets us reason using basic mathematical objects such as
	natural numbers,
	sets, sequences and maps. For example, \texttt{var S: set<nat>;} 
	declares a set of natural numbers and \texttt{S + T} results in the 
	union of these two sets.
	\item Dafny offers a fully automated way of proving, so we do not
		have to worry about which axioms have to be applied where. 
		Instead, it translates the verification task into first-order logic
		and delegates the proof to a SMT solver.\footnote{The downside being
		that it is hard to debug failing proofs.}
\end{itemize}

A manual of Dafny can be found here:
\url{https://github.com/Microsoft/dafny/raw/master/Docs/DafnyRef/out/DafnyRef.pdf}.

Dafny can be executed online on Microsofts Rise4Fun website or 
as a plugin of the free tool Visual Studio Code.
Here is an example \url{https://rise4fun.com/Dafny/Mul}.

A link to the exercise in Rise4Fun will be put on blackboard.


\section{Short manual for Handling the Code}


Extract the zip file from blackboard and make 
sure \ltsmin is still installed.\footnote{In Lab 1,
\ltsmin was installed in \texttt{\$HOME/ltsmin}.
Make sure \texttt{\$HOME/ltsmin/bin} is still in your
\texttt{\$PATH}, i.e., the commands \texttt{pins2lts-mc} and \texttt{spins} are executable.}
The code can be compiled with the following commands:

\texttt{cmake .}

\texttt{make}


To run the model checker call:

\texttt{./mc <model>}

Here, \texttt{<model>} can be a Promela model
from the examples dir (which should be compiled
automatically to a \texttt{.spins} library
provided that \texttt{spins} is in your path).
Alternatively, the Peterson from Lab 1 model can 
still be used. It is compiled to \texttt{libpeterson.so}
in the \texttt{peterson} subdir. It can thus be run with:

\texttt{./mc peterson/libpeterson.so}






\section{Hand in Results}

Write a short report (as short as possible, but including the table asked in the first part of the lab) answering the above questions and zip it together with
your source code.
Hand in your results in the `Labs' section on Blackboard.



\bibliographystyle{plain}
\bibliography{lit}


\end{document}

%%% Local Variables:
%%% mode: latex
%%% TeX-master: t
%%% End:
