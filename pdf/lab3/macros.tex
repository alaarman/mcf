%!TEX root = main.tex


\newcommand\defaccr[2]{\newcommand#1{#2\xspace}}
\newcommand\defmath[2]{\newcommand#1{\ensuremath{#2}\xspace}}
\newcommand\concept[1]{\textit{#1}}



\newcommand\ccode[1]{\texttt{#1}}


\newcommand{\overarrowi}[1]{\xrightarrow{#1}}
\newcommand{\overarrow}[1]{
  \mathchoice{\raisebox{-3pt}{ $\overarrowi{#1}$ }}
             {\raisebox{-3pt}{ $\overarrowi{#1}$ }}
             {\raisebox{-3pt}{ $\overarrowi{#1}$ }}
             {\raisebox{-3pt}{ $\overarrowi{#1}$ }}}

%containers
\providecommand{\tuple}[1]{\ensuremath{\left( #1 \right)}}
\providecommand{\set}[1]{\ensuremath{\left\lbrace #1 \right\rbrace}}
\providecommand{\sequence}[1]{\ensuremath{\left( #1 \right)}}
\providecommand{\sizeof}[1]{\ensuremath{\left\vert{#1}\right\vert}}
\providecommand{\vect}[1]{\ensuremath{( \begin{matrix} #1 \end{matrix} )}}
\providecommand{\always}[1]{\ensuremath{\left[ #1 \right]}}
\providecommand{\possibly}[1]{\ensuremath{\left\langle #1 \right\rangle}}
%\newcommand{\powerset}[1]{\wp({#1})}
\newcommand{\powerset}[1]{\ensuremath{\mathbf{2}^{#1}}}
% Semantics
\newcommand{\eval}[2][]{\ensuremath{\llbracket #2\rrbracket^{#1}}}

% Map (bijection)
\DeclareMathOperator{\bijmap}{%
 \rlap{\ensuremath{\rightarrowtail}}%
 	{\ensuremath{\mkern2mu\twoheadrightarrow}}}

% Abs, Floor, Ceil
\providecommand{\abs}[1]{\lvert#1\rvert}
\providecommand{\floor}[1]{\lfloor#1\rfloor}
\providecommand{\ceil}[1]{\lceil#1\rceil}

\newcommand{\defn}{\,\triangleq\,}

% Rotate text
\newcommand{\turner}[3][10em]{% \turn[<width>]{<angle>}{<stuff>}
  \rlap{\rotatebox{#2}{\begin{varwidth}[t]{#1}#3\end{varwidth}}}%
}


\providecommand{\Assign}[2]{{#1{~:=~}#2}}
\providecommand{\AssignState}[2]{\State \Assign{#1}{#2}}


%initial state
\newcommand{\inits}{s^0}

%transition relation expression
\newcommand{\transrelexpr}{\mathsf{next}}%\mathord{\hookrightarrow}}

\defmath{\explored}{\mathsf{explored}}
\defmath{\visited}{\mathsf{visited}}
\defmath{\level}{\mathsf{level}}
\defmath{\tmp}{\mathsf{tmp}}
\defmath{\WM}{\mathsf{WM}}
\defmath{\RM}{\mathsf{RM}}
%next state small caps
\renewcommand{\next}{\mathsf{next}}

%next state small caps
\newcommand{\nextstate}{\textsc{next-state-r2w}\xspace}
