\documentclass[12pt]{article}

\usepackage{amsmath,amsthm,amssymb}
\usepackage{hyperref}


\begin{document}
	\title{Implementing a DPLL Sat Solver}
	\author{Pim Wijn and Alfons Laarman}
	
	\maketitle
	
	The goal is to implement a sat solver based on the DPLL algorithm that can be used on satisfiability problems of modest size.
	The implementation can be separated in three steps: implementing the backtracking structure of the algorithm, adding boolean constraint propagation and improving the underlying data-structures.
	
	It is not necessary to completely implement all supporting functionality and structures. The code provided contains gaps that have to be filled in to implement the algorithm. Each of the gaps contains a \textbf{TODO} with a short explanation. Extra methods can be added if necessary, but they are not required for the final implementation (if you do so, make sure to document them well).
	
	All relevant header files contain documentation for each non-trivial method.
	
	\paragraph{Resources}
	\begin{enumerate}
		\item SAT and DPLL basics:  \url{http://satassociation.org/articles/sat.pdf}
		\item Lazy data-structures (section 4.5.1) \url{http://satassociation.org/articles/FAIA185-0131.pdf}
	\end{enumerate}
	
	
	\subsubsection*{Step 1: DPLL Backtracking}
	First it is necessary to implement a simple backtracking search. This resulting implementation is not very efficient, but simple. It will help in implementing extensions to the algorithm.
	For the first step it is necessary to implement:
	\begin{enumerate}
		\item a first version of the \textbf{DPLL} function in \path{src/DPLL/DPLL.cpp}
	\end{enumerate}

	\subsubsection*{Step 2: Boolean Constraint Propagation}
	At this point you have a simple algorithm that searches through all possible combinations of assignments. This can be improved by propagating forced literals (unit clauses).
	To do this, implement:
	\begin{enumerate}
		\item \textbf{isUnitClause} in \path{src/structures/Clause.cpp}
		\item \textbf{getImpliedLiteral} in \path{src/structures/Clause.cpp}
		\item \textbf{unitPropagate} in \path{src/DPLL/DPLL.cpp}
		\item improve \textbf{DPLL} to use \textbf{unitPropagate} before choosing the next literal
		\item \textbf{setLiteral} in \path{src/structures/CNF.cpp} (to actively track counters, or later watched literals)
	\end{enumerate}
	
	\subsubsection*{Step 3: Watched Literals (optional)}
	At this point it is still necessary to explicitly evaluate all clauses and literals to find unit clauses.
	This can be prevented by using watched literals and lazy data-structures.
	For this in \path{src/structures/Clause.cpp} implement:
	\begin{enumerate}
		\item Improve \textbf{getImpliedLiteral} to use watched literals
		\item Improve \textbf{isUnitClause} to use watched literals
		\item Implement \textbf{updateWatchedLiteral}
	\end{enumerate}
	In \path{src/structures/CNF.cpp} implement:
	\begin{enumerate}
		\setcounter{enumi}{3}
		\item Add the updating of watched literals to \textbf{setLiteral}
	\end{enumerate}
	
There are numerous of larger and smaller examples in the \texttt{examples/}	 directory (see \texttt{README.txt} for SAT / UNSAT versions). Make sure to test you test your implementation on various examples.
Write a small report where you keep track of the improvements between
the different versions of the algorithm that you created and hand it in on Blackboard.
	
\end{document}}